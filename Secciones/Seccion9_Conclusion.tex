\section{Conclusi�n}
\subsection{Queda claro}

\begin{frame}{Recapitulemos\ldots}
Hemos aprendido\ldots
\begin{itemize}
\item <1-> \alert<1>{Qu� es el \LaTeX{}}
\item <2-> \alert<2>{Tipos de documentos y c�mo estructurar el c�digo}
\item <3-> \alert<3>{Partes del documento: estructura jer�rquica, grandes unidades de estructura, elementos fundamentales \ldots}
\item <4-> \alert<4>{Los tipos de entornos que podemos utilizar y definir}
\item <5-> \alert<5>{C�mo personalizar nuestro documento en cuanto a m�rgenes, tama�o y tipo de fuente, etc.}
\item <6-> \alert<6>{Una breve introducci�n a las matem�ticas}
\item <7-> \alert<7>{Inclusi�n de tablas y gr�ficos}
\end{itemize}

\only<8->{
Tan solo hemos avistado, a lo lejos, la punta del iceberg: La potencia de  \LaTeX{} est� en la modularidad:
\only<9->{
\begin{center}
\sc \Large �Investiga sus paquetes!
\end{center}
}
}
\end{frame}


\subsection{�Y ahora qu�?}

\begin{frame}{Algunos recursos web}
\begin{itemize}
\item <1-> {\color{blue}\url{http://www.ctan.org/search}}: Cat�logo de paquetes de \TeX{}
\item <2-> {\color{blue}\url{http://detexify.kirelabs.org/classify.html}}: Detector de s�mbolos (lo dibujas, lo reconoce y te muestra el comando en  \LaTeX{})
\item <3-> {\color{blue}\url{http://www.latex-project.org/}}: P�gina oficial del proyecto \LaTeX{}
\item <4-> {\color{blue}\url{http://en.wikibooks.org/wiki/LaTeX}}: Resulta de gran utilidad por sus numerosos ejemplos
\item <5-> {\color{blue}\url{http://www.howtotex.com/}}: P�gina con muchos ejemplos
\item <6-> {\color{blue}\url{http://www.latextemplates.com/}}: Plantillas de algunos tipos de documentos
\end{itemize}
\end{frame}





\begin{frame}
\vspace{0.75cm}
\begin{figure}[hbtp]
\centering
\includegraphics[width=0.7\textwidth, page=1]{Figuras/gracias2}
\end{figure}
\end{frame}



